\documentclass[thesis.tex]{subfiles}
\begin{document}

\chapter{Conclusion}\label{chap:concl}

\section{Summary}
This work created an heuristic based approach to efficiently monitor traffic inside a network. The goal of the monitoring was to identify user inside it which overuses their resources with the indent to punish them and to balance the network capacity for all user. This was especially for the case of multi-path communication inside the SCION network structure, which allows to send data over multiple path at the same time and to better utilize the available transfer capacity of the network. 

At first the SCION network structure was described, which stands for a new structure of the Internet itself. Two already existing approaches were considered to load balance the bandwidth usage, SIPRA and the VirtualCredit. The project itself should be applicable for the test bed, SCIONLab, and for this, the SIPRA approach was considered as to complicate to implement. The latter approach was shown, that it is also to complicate to correctly realize without dead ends or possibilities to abuse the system.

A third way was chosen: To probe the networks traffic and identify the user with the most bandwidth. A requirement was to be unpredictable, otherwise the user could avoid the probes. The proposed solution should also be scalable and so use as few probes as possible. It should also be adoptable for the networks circumstances. This approach was named SpeedCam and in this work discussed and evaluated.

The problem of monitoring a unit which has an interest in cheating is not new but can be based on the inspection game, which is part of the game theory. The general concept of SpeedCam is based on this with the inspector being the monitoring component and the users are the inspectees. Based on this game a general concept was developed and discussed, which can be applied to any type of network. The concept consists of four stages with a repeat strategy: The exploration, the selection, the monitoring and the conclusion. A sequence of them is called an episode and over time the inspector tries to learn the best positions to probe the traffic. Each stage can be done with multiple strategies which are discussed and evaluated in this work. 

The general concept was adapted for the SCION structure and implemented in its test bed, SCIONLab. The monitoring was done with an established technology, the Prometheus framework. This should show the precision, the success rate of identifying a greedy user and the performance impact of the approach. The evaluation was done over 48h monitoring the traffic, where a Prometheus server captured the complete network and provided the real traffic inside the network. There were ten different configurations of the SpeedCam implementation running in parallel to show the impact of these different parameters. A problem with this approach was the missing maximum bandwidth information and because of that the hit rate was slightly adapted to identify not a greedy user, but a maximum bandwidth usage. Another problem was that not all nodes exposed their interface to gather the bandwidth information, so that only ~20\% of the nodes were usable as SpeedCams.

\section{Evaluation}
The results showed that the different configurations have an influence of the hit rate, the precision and slightly on the performance itself. It was shown that the precision and the hit rate is primarily based on the inspection interval and secondarily on the amount of probes. The standard interval was 10s between an inspection and multiple configurations used this. The scaling configurations, which used only different amount of nodes but the same interval, had the highest precision and hit rat, but also the highest impact on the performance. On the contrary, the random based strategies had the worst results. The experience based one uses a random time to explore good points for monitoring, but this decision was not good. Using a timespan of a day and a fixed interval to learn the networks behaivor and than to use this experience to decide the time spots could improve the results.

The proposed heuristic based on the networks structure and the previously recorded bandwidth is working in general as seen with the \textit{const} configuration with always only one probe. Its result were surprisingly and it was expected to have worse results. The hit rate for the two highest user was about 50\% and it captured ~15\% of the networks total traffic in average. This will not scale with larger networks and traffic which resides only in parts of them. In SCIONLab, most of the traffic origins from the ISD 1 and is linked to it. But the inspector do not have to monitor the complete network as described in the general concept. Multiple inspectors can be deployed to different parts of the network, for example one per ISD, and only monitor this sub-network. This would counter the problem of the scalability.

The impact on the performance was acceptable and always under 5\% of the machines resources in sight of CPU and memory consumption. This scales with the amount of utilized SpeedCams and the size of the monitored network. The performance could be even improved by optimizing the prototype, for example use a sparse graph matrix instead of a mirrored one. 

The \textbf{1.} of the defined goals (\autoref{sec:intro:goals}) is achieved by providing multiple configurations and evaluating them for the approach. This also meets the \textbf{3.}. The \textbf{2.} was only partially achieved. First the hit rate was differently defined because off the test environment. Second the goal is only achieved for a linear scaling of 20\% of the nodes when allowing the hit also for the second position. These goal has to be proved with a network where the bandwidth capacity is known and a congestion can be detected. The \textbf{4.} was partially achieved by having a test environment with around 100 nodes and a solution with a performance impact under 5\%. The network traffic was not minimized by getting unnecessary information from the Prometheus metrics. From 27KB of information were only three information used. This can be reduced by using other protocols or minimizing the exposed metrics on the border router. The SpeedCam can be applied to other network types and this achieves \textbf{5.}. The implementation is pushed on GitHub and an instruction how to use is written. This solves \textbf{6.} in partial, because the implementation is a prototype and should not be used in its current state in a productive environment. 

All in all showed this work the potential use of SpeedCam to monitor the traffic inside a network.

\section{Future Work}
There are multiple improvements of the proposed work and also next steps what to do. These ideas will be described in this section.

The most important idea would be to measure the hit rate of the inspection as proposed in the general concept. This needs the information about the capacity of a connection. It can be retrieved in SCION from the coordinator\footnote{\url{https://github.com/netsec-ethz/scion-coord}, 15.04.2018}, but at this point of time the information was not reliable enough. There should also be 

The results should be verified over a longer time, for example over a week or a month. This could show the development of the hit rate and the measure The prototype implementation was stable for two days and could be used for longer inspection times. The used machine should have enough disk space, because the raw data of two days were 2,63 GB in size plus the Prometheus server data, which were around 8GB in size for a timespan of ten days. A longer timespan will result in a higher data volume. 

\subfilebib % Makes bibliography available when compiling as subfile
\end{document}