\documentclass[thesis.tex]{subfiles}
\begin{document}

\addchap{Index of Notation}

In a lot of cases it makes sense to give an overview over your mathematical notation.

\subsubsection*{Mathematical}

\begin{tabular}{ l l }
$\mathrm{x}$ & Point in 3D space \\
$\overrightarrow{\mathrm{x}\mathrm{y}}$ & Normalized direction vector from $\mathrm{x}$ to $\mathrm{y}$\\
$\mathbf{v}$ & Direction vector in 3D space \\
$\mathrm{p}_x, \mathbf{v}_x$ & $x$ component of point / vector\\
$\mathbf{v} \cdot \mathbf{w}$ & Dot product of vectors $\mathbf{v}$ and $\mathbf{w}$\\
$(\mathbf{v} \cdot \mathbf{w})^+$ & Dot product of vectors $\mathbf{v}$ and $\mathbf{w}$ with negative values clamped to zero\\
$\mathbf{v} \times \mathbf{w}$ & Cross product of vectors $\mathbf{v}$ and $\mathbf{w}$\\
$||\mathbf{v}||$ & Euclidean length of vector $\mathbf{v}$\\
$\hat{\mathbf{v}}$ & Normalized vector $\mathbf{v}$
\end{tabular}


\subsubsection*{Quantities \& Functions}

\begin{tabular}{ l l l}
$A$ & \textbf{Area}\\
$\omega$ & \textbf{Solid Angle}\\
$\phi$ & \textbf{Radiant Flux}, light power\\
$I$ & \textbf{Radiant Intensity}, flux density per solid angle\\
$E$ & \textbf{Irradiance}, flux density per area\\
$L$ & \textbf{Radiance}, flux density per area per solid angle\\
$\rho$ & \textbf{Reflectance}, ratio between incoming and outgoing flux\\
$f_r$ & \textbf{BRDF}, function on the relation between irradiance and outgoing radiance\\
\end{tabular}


\end{document}